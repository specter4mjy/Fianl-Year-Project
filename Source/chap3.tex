\chapter{Current Source}
\section{Single current source circuit}



\begin{figure}[htbp]
\centering
\includegraphics[scale =0.6]{"../Photo/Chap1/single current source"}
\caption{single current source circuit}
\label{fig:singlecurrentsource}
\end{figure}

Generally, we need a constant current source in circuit and the most classic one shows in Figure \ref{fig:singlecurrentsource}.

As we know the forward voltage cross a diode is about 0.65V which is approximately equal to $ V_{be} of transistor $. 
\[ V_B = 2 \times V_{diode} = V_{be} + V_{R_{33}}  \]   
Therefore:
\[  V_{R_{33}} = V_{diode} = 0.65V \] 
\[ I_e = \frac{V_{R_{33}}}{R_{33}} = \frac{0.65V}{1150 \Omega} = 565.217 \mu A \]

\begin{figure}[htbp]
\centering
\includegraphics[scale=1]{"../Photo/Chap1/single current source simulation result"}
\caption{single current source circuit simulation result}
\label{fig:singlecurrentsourcesimulationresult}
\end{figure}

From Figure \ref{fig:singlecurrentsourcesimulationresult}, we can see simulation result is close to the value we calculated. This simple circuit are able to supply constant current.


\section{Use current source to replace output resistor}

Now we can use current source to replace the  $ R_{16} $  in circuit of Figure \ref{fig:transistorfeedback}.
Current source can supply stable current output. Till this step, we have finished the voltage amplifier part circuit but current of output is still enough to drive a headphone.
 
\begin{figure}[htbp]
	\centering
	\includegraphics[scale=0.6 ]{"../Photo/Chap1/add cuurent source"}
	\caption{the circuit after adding current source}
	\label{fig:addcuurentsource}
\end{figure}