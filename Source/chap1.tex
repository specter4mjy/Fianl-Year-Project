\chapter{Simple transistor circuit}

\section{Transistor basic property}
Figure \ref{fig:singletransistorcircuit} shows the basic NPN bipolar junction transistor circuit.

\begin{figure}[htbp]
\centering
\includegraphics[scale=0.7]{"../Photo/Chap1/single transistor circuit"}
\caption{Single transistor circuit}
\label{fig:singletransistorcircuit}

\end{figure}

 We can get transistor operating state from simulation result as Table \ref{tab:DC operating point analysis result}. It's obvious that $I_C$ and $I_E$ is proximately 200 times greater than $I_B$ which is the main function of transistor.

\begin{table}[h]
	\centering 
	
	\begin{tabular} { >{\columncolor{mycolor} \centering}m{3cm}  | >{\centering\arraybackslash }m{3cm}}
		\hline 
		$I_B$ & $9.09789\mu$ \\ 
		\hline 
		$I_C$ & $2.02293m$ \\  
		\hline 
		$I_E$ & $-2.03003m$ \\  
		\hline 		
	\end{tabular}  
	
	\caption{DC operating point analysis result} 
	\label{tab:DC operating point analysis result}
\end{table}

 Equation \ref{equ:betedefinition} defines $\beta$ which is the most important parameter of transistor.  
\begin{equation}
\beta = \frac{I_C}{I_B}
\label{equ:betedefinition}
\end{equation}


\section{Limit current gain} 
Generally, we need a method to control the current gain as we want. Figure \ref{fig:basictransistorcircuitwithRcandRe} is a simply solution by adding transistor $R_C$ and $R_E$.

\begin{figure}[htbp]
\centering
\includegraphics[scale=0.7]{"../Photo/Chap1/basic transistor circuit with Rc and Re"}
\caption{Basic transistor circuit with $R_c$ and $R_e$}
\label{fig:basictransistorcircuitwithRcandRe}
\end{figure}

We can derive voltage gain $A_V$ with Equation \ref{equ:Avdefinition}. And in circuit in Figure \ref{fig:basictransistorcircuitwithRcandRe}, $A_V$ is approximate 5 theoretically.
\begin{equation}
A_V \triangleq \frac{ V_{out}} { V_{in}} \approx -\frac{R_C}{R_E}
\label{equ:Avdefinition}
\end{equation}

From simulation result in Figure \ref{fig:basictransistorciruitoutput}, the practical $A_V = \frac{7.6486m}{2m}=3.8243$.

\begin{figure}[htbp]
\centering
\includegraphics[scale=0.6]{"../Photo/Chap1/basic transistor ciruit output wave"}\\[0.5cm]
\includegraphics[scale=1]{"../Photo/Chap1/basic transistor ciruit output cursor data"}
\caption{Output of the circuit in Figure \ref{fig:basictransistorcircuitwithRcandRe} }
\label{fig:basictransistorciruitoutput}
\end{figure}

