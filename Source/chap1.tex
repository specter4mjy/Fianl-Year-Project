\chapter{Simple transistor circuit}

Figure \ref{fig:singletransistorcircuit} shows the basic NPN bipolar junction transistor circuit that base voltage makes transistor fully switch on.

\begin{figure}[htbp]
\centering
\includegraphics[scale=0.7]{"../Photo/Chap1/single transistor circuit"}
\caption{Single transistor circuit}
\label{fig:singletransistorcircuit}

\end{figure}


\begin{table}[htbp]
	\centering 
	
	\begin{tabular} { >{\columncolor{mycolor} \centering}m{3cm}  | >{\centering\arraybackslash }m{3cm}}
		\hline 
		$I_B$ & $9.09789\mu$ \\ 
		\hline 
		$I_C$ & $2.02293m$ \\  
		\hline 
		$I_E$ & $-2.03003m$ \\  
		\hline 		
	\end{tabular}  
	
	\caption{DC operating point analysis result} 
	\label{tab:DC operating point analysis result}
\end{table}







Therefore, we need a method  which we can control the output current with, see Figure \ref{fig:basictransistorcircuitwithRcandRe}.

\begin{figure}[htbp]
\centering
\includegraphics[scale=0.7]{"../Photo/Chap1/basic transistor circuit with Rc and Re"}
\caption{Basic transistor circuit with $R_c$ and $R_e$}
\label{fig:basictransistorcircuitwithRcandRe}
\end{figure}
