\chapter{Simple transistor circuit}

\section{Transistor basic property}
Figure \ref{fig:singletransistorcircuit} shows the basic NPN bipolar junction transistor circuit.

\begin{figure}[htbp]
\centering
\includegraphics[scale=0.6]{"../Photo/Chap1/single transistor circuit"}
\caption{Single transistor circuit}
\label{fig:singletransistorcircuit}

\end{figure}

 We can get transistor operating state from simulation result as Figure \ref{fig:single transistor circuit data tablet}. It's obvious that $I_C$ and $I_E$ is proximately 200 times greater than $I_B$ which is the main function of transistor.

 \begin{figure}[htbp]
 	\centering
 	\includegraphics[scale=1]{"../Photo/Chap1/single transistor circuit data table"}
 	\caption{Single transistor circuit simulation data}
 	\label{fig:single transistor circuit data tablet}
 	
 \end{figure}

 Equation \ref{equ:betedefinition} defines $\beta$ which is the most important parameter of transistor.  
\begin{equation}
\beta = \frac{I_C}{I_B}
\label{equ:betedefinition}
\end{equation}

\section{Find $ R_e $ }

\begin{figure}[htbp]
  \centering
  \includegraphics[scale=0.6]{"../Photo/Chap1/Ie is 2 mA Vbe is 657mV"}\\[0.5cm]
  \includegraphics[scale=0.8]{"../Photo/Chap1/Ie is 2 mA Vbe is 657mV data"}
  \caption{$ V_{be} $ and $ I_c $ curve}
  \label{fig:VbeandIccurve}
\end{figure}

After running DC sweep command on $ V4 $ in circuit of Figure \ref{fig:singletransistorcircuit}, We can get the curve of Figure \ref{fig:VbeandIccurve}. This illustrate that when $ V_be = 657.7 mV $, $ I_e = 2 mA $.

\begin{figure}[htbp]
	\centering
	\includegraphics[scale=0.6]{"../Photo/Chap1/Re model"}\\[0.5cm]
	\includegraphics[scale=0.8]{"../Photo/Chap1/Re model data"}
	\caption{Re model}
	\label{fig:Re model}
\end{figure}

If we zoom in Figure \ref{fig:VbeandIccurve} like shown in Figure \ref{fig:Re model}, the relationship between $ V_{be} $ and $ I_e $ is linear which is same as resistor and we called $ R_e $.Then we can get its value with
\[ R_e = \frac{dx}{dy} = \frac{143.472u}{10.66u} = 13.459 \Omega\]
Because we know $ I_e = 2mA $, 
\[ Re = \frac{V}{I_e} = \frac{V}{2mA}=13.459\Omega\]
\[ V = 2mA \times 13.459 \approxeq 26mV \] 
Therefore,we can calculate $ R_{e} $ with $ I_e $ in future using
\begin{equation}
	R_{e} = \frac{26 mV}{I_{e}}
\end{equation}


\section{Limit current gain} 
Generally, we need a method to control the current gain as we want. Figure \ref{fig:basictransistorcircuitwithRcandRe} is a simply solution by adding transistor $R_C$ and $R_E$.

\begin{figure}[htbp]
\centering
\includegraphics[scale=0.6]{"../Photo/Chap1/basic transistor circuit with Rc and Re"}
\caption{Basic transistor circuit with $R_c$ and $R_e$}
\label{fig:basictransistorcircuitwithRcandRe}
\end{figure}

We can derive voltage gain $A_V$ with Equation \ref{equ:Avdefinition}. And in circuit in Figure \ref{fig:basictransistorcircuitwithRcandRe}, $A_V$ is approximate 5 theoretically.
\begin{equation}
A_V \triangleq \frac{ V_{out}} { V_{in}} \approx -\frac{R_C}{R_E}
\label{equ:Avdefinition}
\end{equation}



\begin{figure}[htbp]
\centering
\includegraphics[scale=0.5]{"../Photo/Chap1/basic transistor ciruit output wave"}\\[0.5cm]
\includegraphics[scale=1]{"../Photo/Chap1/basic transistor ciruit output cursor data"}
\caption{Output of the circuit in Figure \ref{fig:basictransistorcircuitwithRcandRe} }
\label{fig:basictransistorciruitoutput}
\end{figure}

From simulation result in Figure \ref{fig:basictransistorciruitoutput}, the practical $A_V = \frac{7.6486m}{2m}=3.8243$ which is close to theoretic value.

\section{Add voltage divider}

As we know, we need make sure $ V_{be} > 0.65V $ for transistor operating correctly. But in practical application, it's hard to keep input signal always meeting this requirement. So we can add capacitor and voltage divider solve this problem like Figure \ref{fig:add voltage divider}. In which, capacitor block the original DC voltage of input signal and voltage divider add the DC voltage which we require to signal. Finally, we use another capacitor for outputting pure AC signal form our circuit. 

\begin{figure}[htbp]
	\centering
	\includegraphics[scale=0.6]{"../Photo/Chap1/add voltage divider"} 
	\caption{Add voltage divider and capacitors }
	\label{fig:add voltage divider}
\end{figure}



