%!TEX root = main.tex
\chapter{Output Stage}

\section{Class A Output Stage}

In Figure \ref{fig:ClassAoutputstage}, transistor Q2 and resistor R1 made up of Class A output stage.

\begin{figure}[htbp]
\centering
\includegraphics[scale=0.6]{"../Photo/Chap4/Class A output stage"}
\caption{Class A output stage}
\label{fig:ClassAoutputstage}
\end{figure}


\begin{figure}[htbp]
\centering
\includegraphics[scale=0.7]{"../Photo/Chap4/Class A simulation result"}
\caption{Class A output stage simulation result}
\label{fig:ClassAsimulationresult}
\end{figure}

As we can see in Figure \ref{fig:ClassAsimulationresult}, output signal of class A output stage is good enough to follow the input signal.

\begin{figure}[htbp]
\centering
\includegraphics[scale=1]{"../Photo/Chap4/Class A DC operating current"}
\caption{Class A DC operating current simulation data}
\label{fig:ClassADCoperatingcurrent}
\end{figure}

But from DC operating simulation result in Figure \ref{fig:ClassADCoperatingcurrent} we know Class A will consume a lot of current from battery and resistor R1 also waste a lot of power. It's can't acceptable because the final circuit is powered by battery.


\section{Class AB Output Stage}

For better efficiency, we tried Class AB output stage which showed in Figure \ref{fig:ClassABoutputstage}. Transistor Q4 and Q5 are used for amplifying upper and lower part of input signal. Resistor R10 and R11 and transistor Q7 are made up of $ V_{be}  $ multiplier. In this case, it generate 2$ V_{be} $ cross between collector and emitter of Q7 which eliminates the crossover distortion caused by $ V_{be}  $ of Q4 and Q5.

\begin{figure}[htbp]
\centering
\includegraphics[scale=0.6 ]{"../Photo/Chap4/Class AB output stage"}
\caption{Class AB output stage}
\label{fig:ClassABoutputstage}
\end{figure}



\begin{figure}[htbp]
\centering
\includegraphics[scale=0.7 ]{"../Photo/Chap4/Class AB output stage simlation result"}
\caption{Class AB output stage simulation result}
\label{fig:ClassABoutputstagesimlationresult}
\end{figure}

Finally, we can see from Figure \ref{fig:ClassABoutputstagesimlationresult} there is almost no distortion in output signal.

\begin{figure}[htbp]
\centering
\includegraphics[scale=1 ]{"../Photo/Chap4/Class AB output stage DC operatingh current data"}
\caption{Class AB output stage DC operating current}
\label{fig:ClassABoutputstageDCoperatinghcurrentdata}
\end{figure}

In Figure \ref{fig:ClassABoutputstageDCoperatinghcurrentdata}, we know that DC operating current of Class AB output stage is much smaller than Class A. Therefore, Class AB output stage much more efficient and meet our requirement.
