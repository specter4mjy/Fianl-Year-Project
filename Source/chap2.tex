%!TEX root = main.tex
\chapter{Negative Feedback}
\section{Simple Negative Feedback system}

\begin{figure}[htbp]
	\centering
	\includegraphics[scale=0.7]{"../Photo/Chap2/Feedback system"}
	\caption{simple negative feedback system}
	\label{fig:Feedback system}
\end{figure}

Figure \ref{fig:Feedback system} show a simple negative feedback  system in \cite{Negativefeedback}which A is ideal amplifier and B is feedback network. 

\[  \frac{V_{out}}{V_{in}} = \frac{1}{B}  \] 
 



\section{Implement Using Op-amp}

In Figure \ref{fig:Op-amp feedback}, an Op-amp 741 is used to implement the negative feedback circuit in Figure \ref{fig:Feedback system}.

\begin{figure}[htbp]
	\centering
	\includegraphics[scale=0.7]{"../Photo/Chap2/Op-amp feedback"}
	\caption{Implement negative feedback circuit with Op-amp}
	\label{fig:Op-amp feedback}
\end{figure}




741 is part A while R1 and R2 form feedback network. 
\[ \frac{V_{out}}{V_{in}} = \frac{1}{B} = \frac{R1 + R2}{R2} = \frac{10K + 1K}{1K} = 11 \]
 
\begin{figure}[htbp]
	\centering 
	\includegraphics[scale=0.6]{"../Photo/Chap2/Op-amp feedback simulation wave"}\\[0.5cm]
	\includegraphics[scale =1]{"../Photo/Chap2/Op-amp feedback simulation cursor data"}
	\caption{Op-amp  feedback simulation result curve}
	\label{fig:Op-amp  feedback simulation result }
\end{figure}

From simulation result in Figure \ref{fig:Op-amp  feedback simulation result }, 
\[ \frac{1}{B} = \frac{\Delta V_{out}}{\Delta V_{in}} =  \frac{2.1972}{199.8674m} =  10.99328 \]
 
Obviously, the simulation result is very close to our estimation.

\section{Find the function of Feedback }

At the beginning, our circuit in Figure~\ref{fig:Feedback init} used is same as the circuit in Figure \ref{fig:Op-amp feedback}, the out put signal is a smooth Sin wave shown as Figure~\ref{fig:Feedback init data}.

\begin{figure}[htbp]
	\centering
	\includegraphics[scale=0.65]{"../Photo/Chap2/Feed_back_ini"}
	\caption{Feedback initial circuit  }
	\label{fig:Feedback init}
\end{figure}

\begin{figure}[htbp]
	\centering
	\includegraphics[scale=0.7]{"../Photo/Chap2/Feedback_ini_data"}
	\caption{Feedback initial circuit output}
	\label{fig:Feedback init data}
\end{figure}

Next we introduce some distortion to the output signal using  pull-push output part. As expect, we could obviously observe crossover distortion for output wave shown in Figure~\ref{fig:Feedback before data}.

\begin{figure}[htbp]
	\centering
	\includegraphics[scale=0.7]{"../Photo/Chap2/Feed_back_ini_before"}
	\caption{Feedback circuit adding pull-push part}
	\label{fig:Feedback before}
\end{figure}

\begin{figure}[htbp]
	\centering
	\includegraphics[scale=0.7]{"../Photo/Chap2/Feedback_ini_before_data"}
	\caption{Feedback circuit adding pull-push part output}
	\label{fig:Feedback before data}
\end{figure}

Next step, we move feedback point from Pin 5 of Op-Amp U6 to the emitter terminal of transistor Q11 as show in Figure~\ref{fig:Feedback after}. From Figure~\ref{fig:Feedback after data}, there's great improvement and we can hardly see any distortion of output signal. It proved that feedback is very useful in aspect of eliminating output distortion.


\begin{figure}[htbp]
	\centering
	\includegraphics[scale=0.7]{"../Photo/Chap2/Feed_back_ini_after"}
	\caption{Feedback circuit after moving feedback point}
	\label{fig:Feedback after}
\end{figure}

\begin{figure}[htbp]
	\centering
	\includegraphics[scale=0.7]{"../Photo/Chap2/Feedback_ini_after_data"}
	\caption{Feedback circuit after moving feedback point output}
	\label{fig:Feedback after data}
\end{figure}


\section{Implement Using transistor}


\begin{figure}[htbp]
\centering
\includegraphics[scale=0.7]{"../Photo/Chap2/transistor feedback"}
\caption{Implement negative feedback circuit with transistor}
\label{fig:transistorfeedback}
\end{figure}

In Figure \ref{fig:transistorfeedback} circuit, Op-amp replaced by circuit in Figure \ref{fig:add voltage divider}. R12 and R15 make up feedback network which $ \frac{1}{B} = \frac{R15+R12}{R12} = 11 $. As we see in Figure \ref{fig:Op-amp  feedback simulation result }, the output is reverse to input. Therefore, we add another transistor Q6 to eliminate the phase difference of signal.


\begin{figure}[htbp]
\centering
\includegraphics[scale=0.6]{"../Photo/Chap2/transistor feedback simulation wave"}\\[0.5cm]
\includegraphics[scale=1]{"../Photo/Chap2/transistor feedback simulation cursor data"}
\caption{transistor feedback circuit simulation result}
\label{fig:transistorfeedbacksimulationwave}
\end{figure}

Apparently, there's no phase difference between input and output signal. 
The voltage gain of circuit in Figure \ref{fig:transistorfeedback} is $ Gain = \frac{\Delta V_{out}}{\Delta V_{in}} = \frac{2.0534}{198.8221m} = 10.3707$. It's also very close to theory result.
 
 
\section{AB calculation}
We know
\[ I_{e1} =\frac{V_{be}}{R_9}=\frac{0.65V}{3300}=0.2mA\]
So
\[ A1=\frac{R_c//R_o}{R_e}=\frac{33000//\frac{75V}{I_e1}}{\frac{25mV}{I_{e1}}}=\frac{3300//\frac{75V}{0.2mA}}{\frac{25mV}{0.2mA}}=26\]

We know
\[ V_{R15}=I_{e1}\times R15=2V\]
\[V_{c2}=V_{b1}-V_{be}-V_{R15}=4.7V\]
\[I_{e2}=I_{c2}=\frac{V_{c2}}{R16}=0.2mA\]
So
\[ A2=\frac{R_c//R_o}{R_e}=\frac{2200//\frac{75V}{I_{e2}}}{\frac{25mV}{I_{e2}}
}=\frac{2200//\frac{75V}{0.2mA}}{\frac{25mV}{0.2mA}}=17.49\]
\[A=A1\times A2=442\]
We know the $B=\frac{1}{11}$
Therefore
\[AB=\frac{442}{11}=40\]
Because AB is less than 100, this is not good enough.