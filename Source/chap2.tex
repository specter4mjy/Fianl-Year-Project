%!TEX root = main.tex
\chapter{Negative Feedback}
\section{Simple Negative Feedback system}

\begin{figure}[htbp]
	\centering
	\includegraphics[scale=0.6]{"../Photo/Chap2/Feedback system"}
	\caption{simple negative feedback system}
	\label{fig:Feedback system}
\end{figure}

Figure \ref{fig:Feedback system} show a simple negative feedback system in which A is ideal amplifier and B is feedback network. 

\[  \frac{V_{out}}{V_{in}} = B  \] 

\section{Implement Using Op-amp}


\begin{figure}[htbp]
	\centering
	\includegraphics[scale=0.6]{"../Photo/Chap2/Op-amp feedback"}
	\caption{Implement negative feedback circuit with Op-amp}
	\label{fig:Op-amp feedback}
\end{figure}


In Figure \ref{fig:Op-amp feedback}, an Op-amp 741 is used to implement the negative feedback circuit in Figure \ref{fig:Feedback system}.

741 is part A while R1 and R2 form feedback network. 
\[ \frac{V_{out}}{V_{in}} = B = \frac{R1 + R2}{R2} = \frac{10K + 1K}{1K} = 11 \]
 
\begin{figure}[htbp]
	\centering 
	\includegraphics[scale=0.6]{"../Photo/Chap2/Op-amp feedback simulation wave"}\\[0.5cm]
	\includegraphics[scale =1]{"../Photo/Chap2/Op-amp feedback simulation cursor data"}
	\caption{Op-amp  feedback simulation result curve}
	\label{fig:Op-amp  feedback simulation result }
\end{figure}

From simulation result in Figure \ref{fig:Op-amp  feedback simulation result }, 
\[ B = \frac{\Delta V_{out}}{\Delta V_{in}} =  \frac{2.1972}{199.8674m} =  10.99328 \]
 
Obviously, the simulation result is very close to our estimation.


\section{Implement Using transistor}


\begin{figure}[htbp]
\centering
\includegraphics[scale=0.6]{"../Photo/Chap2/transistor feedback"}
\caption{Implement negative feedback circuit with transistor}
\label{fig:transistorfeedback}
\end{figure}

In Figure \ref{fig:transistorfeedback} circuit, Op-amp replaced by circuit in Figure \ref{fig:add voltage divider}. R12 and R15 make up feedback network which $ B = \frac{R15+R12}{R12} = 11 $. As we see in Figure \ref{fig:Op-amp  feedback simulation result }, the output is reverse to input. Therefore, we add another transistor Q6 to eliminate the phase difference of signal.


\begin{figure}[htbp]
\centering
\includegraphics[scale=0.6]{"../Photo/Chap2/transistor feedback simulation wave"}\\[0.5cm]
\includegraphics[scale=1]{"../Photo/Chap2/transistor feedback simulation cursor data"}
\caption{transistor feedback circuit simulation result}
\label{fig:transistorfeedbacksimulationwave}
\end{figure}

Apparently, there's no phase difference between input and output signal. 
The voltage gain of circuit in Figure \ref{fig:transistorfeedback} is $ Gain = \frac{\Delta V_{out}}{\Delta V_{in}} = \frac{2.0534}{198.8221m} = 10.3707$. It's also very close to theory result.
 
 
